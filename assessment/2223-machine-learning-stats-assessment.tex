\documentclass[a4paper]{tufte-handout}

\usepackage{ians_notes}

\title{Assessment: Machine Learning and Statistics}
\author{ian.mcloughlin@atu.ie}
\date{Winter 22/23}

\begin{document}
 
\maketitle

These are the instructions for the assessment of Machine Learning and Statistics in Winter 22/23.
These cover the full 100\% of the assessment for this module.


\section{Submission}

\begin{itemize}
  \item The deadline for submission is January 7\textsuperscript{th}, 2023. 
  \item Your whole submission must be in a single GitHub repository.
  \item Use the form on the Moodle page to submit your repository.
  \item All you need to do is submit the repository URL.
  \item You should submit the URL as soon as possible.
  \item Commits in GitHub on or before the deadline will be considered.\footnote{Once you have submitted your URL, you do not need to do anything other than commit to your repository and push the changes to GitHub.}
\end{itemize}


\section{What to submit}
This assessment has three overlapping components, as follows.

\newthought{Presentation of GitHub repository $(20\%)$:}
\begin{itemize}
  \item Create a GitHub repository of a standard presentable in job interviews.
  \item Include an informative and concise README.
  \item Organize your repository --- no unnecessary files or clutter.
  \item Work regularly, adding regular reasonably-sized commits.
\end{itemize}

\newthought{Statistics Notebook $(40\%)$:}
\begin{itemize}
  \item Include all your practical work in your JupyterLite instance.
  \item The lecture notes suggest concepts you should explore\footnote{The lecture notes themselves are presented in Jupyter notebooks.}.
  \item You should complete these in notebooks each week\footnote{You don't need to have a notebook for each individual small topic. It is up to you to choose how you organize the notebooks. Aim for one notebook per topic.}.
  \item Plots explaining main concepts.
  \item Code demonstrating key algorithms.
  \item Well-written explanations\footnote{Pitch all your work at your classmates as the audience.}.
\end{itemize} 

\newthought{Machine Learning Notebook $(40\%)$:}
\begin{itemize}
  \item Create a repository for a custom JupyterLite instance.
  \item Add a GitHub action to publish the JupyterLite instance.
  \item Commits to the main branch should trigger the action.
  \item Detail all of this in your README.
\end{itemize}


\section{Marking Scheme}
Each component will be marked using the four categories below.
To receive a good mark in a category, your submission needs to provide evidence of meeting each of the criteria listed under it\footnote{In line with ATU policy, the examiners' overall impression of the submission may affect individual marks in each category.}.

\begin{description}
  \item[Research $(25\%)$:] evidence of research on topics; appropriate referencing; building on work of others; comparison to similar work.
  \item[Development $(25\%)$:] clear, concise, and correct code; appropriate tests; demonstrable knowledge of different approaches and algorithms; clean architecture.
  \item[Documentation $(25\%)$:] clear explanations of concepts in notebooks; concise comments in code and elsewhere; appropriate, standard README for a GitHub repository.
  \item[Consistency $(25\%)$:] tens of commits, each representing a reasonable amount of work; literature, documentation, and code evidencing work on the assessment; evidence of reviewing and refactoring.
\end{description}


\section{Advice}

\begin{itemize}
  \item Students sometimes struggle with the freedom given in an open-style assessment.
  \item You must decide where and how to start, what is relevant content for your submission, how much is enough, and how to make the submission your own.
  \item This is by design --- we assume you have a reasonable knowledge of programming and an ability to source your own information.
  \item Companies tell us they want graduates who can (within reason) take initiative, work independently, source information, and make design decisions without needing to ask for help.
  \item The point of this assessment is to demonstrate you can do that.
  \item You need a plan, you cannot just start coding straight away.
\end{itemize}


\section{Policies}

\begin{itemize}
  \item You are bound by all ATU policies and any GMIT policies that have not yet been replaced by new ATU policies.
  \item Review the GMIT Quality Assurance Framework~\cite{gmitqaf}.
  \item Pay particular attention to the Policy on Plagiarism and the Code of Student Conduct.
  \item If you have any doubts about what is permissible, email me to ask\footnote{\url{ian.mcloughlin@atu.ie}}.
\end{itemize}


\bibliographystyle{plainnat}
\nobibliography{bibliography}

\end{document}